\documentclass[12pt]{article}

%Importationd e toutes les balises n�cessaires
\usepackage[francais]{babel}
\usepackage[latin1]{inputenc}
\usepackage[T1]{fontenc}
\usepackage{graphicx}
\usepackage{moreverb}
\usepackage{listings}
\usepackage{color}
\usepackage{xcolor}

%Creationd es couleurs
\definecolor{dkgreen}{rgb}{0,0.6,0}
\definecolor{mauve}{rgb}{0.58,0,0.82}

%Definition du code couleur des langages
\lstset{breaklines=true, commentstyle=\color{dkgreen}, stringstyle=\color{mauve} , tabsize=3}

\lstdefinestyle{Bash}
{language=bash,
keywordstyle=\color{blue},
basicstyle=\ttfamily,
morekeywords={peter@kbpet},
alsoletter={:~$},
morekeywords=[2]{peter@kbpet:},
keywordstyle=[2]{\color{red}},
literate={\$}{{\textcolor{red}{\$}}}1 
         {:}{{\textcolor{red}{:}}}1
         {~}{{\textcolor{red}{\textasciitilde}}}1,
}
%Fin initialisation du document


\begin{document}

\begin{titlepage}

\newcommand{\HRule}{\rule{\linewidth}{0.5mm}} 

\center 


\textsc{\LARGE Universit� de Caen Basse-Normandie}\\[1.5cm]  %Nom de l'universit�
\textsc{\Large Master DNR2I}\\[0.5cm]  %Nom de la formation
\textsc{\large JAVA}\\[0.5cm] %Nom de la mati�re


\HRule \\[0.4cm]
{ \huge \bfseries Devoir Maison}\\[0.4cm] %Type du devoir
\HRule \\[1.5cm]
 


\begin{minipage}{0.4\textwidth}
\begin{flushleft} \large
\emph{Auteur:}\\
Antoine \textsc{LEBEL} %1uteur
Clement \textsc{LE BIEZ} %1uteur
Guillaume \textsc{BROSSE} %1uteur
Nicolas \textsc{BELLEME} %1uteur
\end{flushleft}
\end{minipage}
~
\begin{minipage}{0.4\textwidth}
\begin{flushright} \large
\emph{Enseignant:} \\
M Y. \textsc{MATHET}  %Enseignant
\end{flushright}
\end{minipage}\\[3cm]

\includegraphics[width=4cm,height=4cm,keepaspectratio]{res/fac.png}\\[0cm] % Include a department/university logo - this will require the graphicx package

{\large \today}\\[3cm] 


\vfill % Fill the rest of the page with whitespace

\end{titlepage}



\tableofcontents % Afficher table des mati�res
\newpage

\part{Question 1}
\section{Intitul� : } T�l�charger le fichier des pr�noms.
\section{R�ponse : }  Fait.

\newpage

\begin{lstlisting}[style=Bash] %Pour mettre du code

generateRandomNumber(){
    number="0";
    for i in {0..8}; do
        r=$((RANDOM%10+0));
        number="$number""$r";
    done
    echo $number;
}

\end{lstlisting}
\end{document}